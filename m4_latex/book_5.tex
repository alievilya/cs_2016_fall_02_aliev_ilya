\documentclass[12pt]{article}
%% Language and font encodings 
\usepackage[english, russian]{babel} 
\usepackage[utf8x]{inputenc} 
\usepackage[T2A]{fontenc}  
\usepackage{fancyhdr} 
%% Sets page size and margins 
 %% Useful packages
\usepackage{setspace} 
\usepackage{lmodern}
\usepackage[a4paper,top=3cm,bottom=2cm,left=3.7cm,right=2.1cm,marginparwidth=1.3cm]{geometry} 
\linespread{1.2}
\begin{document} 
\section{{\bf Dimensions}}

\begin{quote}
\end{quote}

\subparagraph{1.1 Economics: The power of multinational corporations 1}
\subparagraph{1.2 Newtonian mechanics: Free fall 3}
\subparagraph{1.3 Guessing integrals 7}
\subparagraph{1.4 Summary and further problems 11}

\begin{quote}

Our first street-fighting tool is dimensional analysis or, when abbreviated,
dimensions. To show its diversity of application, the tool is introduced
with an economics example and sharpened on examples from Newtonian
mechanics and integral calculus.
\end{quote}
\begin{center}
{\bf \subsection{Economics: The power of multinational corporations}}
\end{center}
Critics of globalization often make the following comparison [25] to prove
the excessive power of multinational corporations:

\begin{quote}
\textit{Nigeria, a relatively economically strong country, the GDP [gross domestic
product] is 99 billion. The net worth of Exxon is 119 billion. “When multinationals
have a net worth higher than the GDP of the country in which they
operate, what kind of power relationship are we talking about?” asks Laura
Morosini.
operate, what kind of power relationship are we talking about?  asks Laura
Morosini.}
\end{quote} 
Before continuing, explore the following question:\\
 {\it What is the most egregious fault in the comparison between Exxon and Nigeria?} \\
The field is competitive, but one fault stands out. It becomes evident after
unpacking the meaning of GDP. A GDP of \$99 billion is shorthand for
a monetary flow of \$99 billion per year. A year, which is the time for
the earth to travel around the sun, is an astronomical phenomenon that
\newpage
\pagestyle{fancy} 
\fancyhf{}
\fancyhead[LO]{\large \textsl{1 Dimensions}} 
\fancyhead[RO]{\large \textsl {\textbf{2}}} 

has been arbitrarily chosen for measuring a social phenomenon—namely,
monetary flow.

Suppose instead that economists had chosen the decade as the unit of
time for measuring GDP. Then Nigeria’s GDP (assuming the flow remains
steady from year to year) would be roughly \$1 trillion per decade and
be reported as \$1 trillion. Now Nigeria towers over Exxon, whose puny
assets are a mere one-tenth of Nigeria’s GDP. To deduce the opposite
conclusion, suppose the week were the unit of time for measuring GDP.
Nigeria’s GDP becomes \$2 billion per week, reported as \$2 billion. Now
puny Nigeria stands helpless before the mighty Exxon, 50-fold larger than
Nigeria.

A valid economic argument cannot reach a conclusion that depends on
the astronomical phenomenon chosen to measure time. The mistake lies
in comparing incomparable quantities. Net worth is an amount: It has
dimensions of money and is typically measured in units of dollars. GDP,
however, is a flow or rate: It has dimensions of money per time and
typical units of dollars per year. (A dimension is general and independent
of the system of measurement, whereas the unit is how that dimension is
measured in a particular system.) Comparing net worth to GDP compares
a monetary amount to a monetary flow. Because their dimensions differ,
the comparison is a category mistake [39] and is therefore guaranteed to
generate nonsense.

\begin{quote}
\textbf{Problem 1.1 Units or dimensions?} 

Are meters, kilograms, and seconds units or dimensions? What about energy,
charge, power, and force?
\end{quote}


A similarly flawed comparison is length per time (speed) versus length:
“I walk 1.5 m s-1—much smaller than the Empire State building in New
York, which is 300 m high.” It is nonsense. To produce the opposite but
still nonsense conclusion, measure time in hours: “I walk 5400 m/hr—
much larger than the Empire State building, which is 300 m high.”

I often see comparisons of corporate and national power similar to our
Nigeria–Exxon example. I once wrote to one author explaining that I
sympathized with his conclusion but that his argument contained a fatal
dimensional mistake. He replied that I had made an interesting point
but that the numerical comparison showing the country’s weakness was
stronger as he had written it, so he was leaving it unchanged!
\newpage
\pagestyle{fancy} 
\renewcommand{\headrulewidth}{0pt} 
\fancyhf{}
\fancyhead[LO]{\large \textsl {\textbf{3}}} 
\fancyhead[RO]{\large \textsl{1.2 Newtonian mechanics}} 

{\bf\subsection{ Newtonian mechanics: Free fall }

A dimensionally valid comparison would compare like with like: either

whereas corporate revenues are widely available, try comparing Exxon’s
annual revenues with Nigeria’s GDP. By 2006, Exxon had become Exxon
Mobil with annual revenues of roughly 350 billion -- almosttwice Nigeria's 2006 GDP of   200 billion. This valid comparison is stronger than the
flawed one, so retaining the flawed comparison was not even expedient!

That compared quantities must have identical dimensions is a necessary
condition for making valid comparisons, but it is not sufficient. A costly
illustration is the 1999 Mars Climate Orbiter (MCO), which crashed into
the surface of Mars rather than slipping into orbit around it. The cause,
according to the Mishap investigation Board, was a mismatch between English and metric units
\begin{quote} The MCO MIB has determined that the root cause for the loss of the MCO
spacecraft was the failure to use metric units in the coding of a ground
software file, Small Forces, used in trajectory models. Specifically, thruster
performance data in English units instead of metric units was used in the
software application code titled SM\_FORCES. A file called Angular Momentum Desaturation \(AMD\) contained the output data from the
SM\_FORCES software. The data in the AMD file was required to be in metric
units per existing software interface documentation, and the trajectory modellers assumed the data was provided in metric units per the requirements
\end{quote}
Make sure to mind your dimensions and units.

Problem 1.2 Finding bad comparisons

Look for everyday comparisons for example, on the news, in the newspaper,
or on the Internet that are dimensionally faulty.
1.2 Newtonian mechanics: Free fall
Dimensions are useful not just to debunk incorrect arguments but also to
generate correct ones. To do so, the quantities in a problem need to have
dimensions. As a contrary example showing what not to do, here is how
many calculus textbooks introduce a classic problem in motion:
\begin{quote}
A ball initially at rest falls from a height of h feet and hits the ground at a
speed of v feet per second. Find v assuming a gravitational acceleration of g
feet per second squared and neglecting air resistance.
\end{quote}
The units such as feet or feet per second are highlighted in boldface
because their inclusion is so frequent as to otherwise escape notice, and
their inclusion creates a significant problem. Because the height is h
feet, the variable h does not contain the units of height: h is therefore
dimensionless. (For h to have dimensions, the problem would instead
state simply that the ball falls from a height h; then the dimension of
length would belong to h.) A similar explicit specification of units means
that the variables g and v are also dimensionless. Because g, h, and v
are dimensionless, any comparison of v with quantities derived from g
and h is a comparison between dimensionless quantities. It is therefore
always dimensionally valid, so dimensional analysis cannot help us guess
the impact speed.

Giving up the valuable tool of dimensions is like fighting with one hand
tied behind our back. Thereby constrained, we must instead solve the
following differential equation with initial conditions:

$$\frac{d^{2}y}{dt^{2}}=-g,\qquad with\ y(0) = h\ and\ dy/dt = 0\ at\ t = 0, \qquad (1.1)$$\\

\pagestyle{fancy} 
\renewcommand{\headrulewidth}{0pt} 
\fancyhf{}
\fancyhead[LO]{\large \textsl {\textbf{4}}} 
\fancyhead[RO]{\large \textsl{1 Dimensions}} 
where y(t) is the ball’s height, dy/dt is the ball’s velocity,  and g is the
gravitational acceleration.\\

Problem 1.3 Calculus solution \\\

Use calculus to show that the free-fall differential equation d2y/dt2 = -g with
initial conditions y(0) = h and dy/dt = 0 at t = 0 has the following solution:
$$\frac{dy}{dt}=-gt,\qquad and\ y = -\frac{1}{2}gt^{2}+h . \qquad  (1.2) $$

\begin{quote} {\it Using the solutions for the ball’s position and velocity in Problem 1.3, what is
the impact speed? }
\end{quote}
When
y(t) = 0, the ball meets the ground. Thus the impact time $t_{0}$ is
$\sqrt{2h/g}$. \\
The impact velocity is −g$t_{0}$ or −$\sqrt{2gh}$. Therefore the impact
speed (the unsigned velocity) is $\sqrt{2gh}$.
This analysis invites several algebra mistakes: forgetting to take a square
root when solving for t0, or dividing rather than multiplying by g when
finding the impact velocity. Practice—in other words, making and correcting many mistakes—reduces their prevalence in simple problems, but
complex problems with many steps remain minefields. We would like
less error-prone methods.


\end{document}
